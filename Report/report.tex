\documentclass{article}
\usepackage{fullpage}
\usepackage{hyperref}

\begin{document}
%Title
\title{Advanced Computer Graphics Final Project Report}
\author{Auston Sterling: sterla@rpi.edu \\ Brendon Justin: justib@rpi.edu}
\date{}
\maketitle

% Your report should describe the technical details of your project; in particular, we want to know about:
%     Any algorithms or data structures you implemented.
%     The core features of your assignment and how you tested them.
%     The challenges that you overcame (or failed to overcome). Note: As you're working on your project, save "blooper" images or video that show your intermediate results and how you debugged your project, and include them in your report & presentation.
%     Images/screenshots/visualizations/video showing the results of your project. Include simple and moderately complex examples.
%     Any known bugs or limitations in your implementation, and potential avenues for future work.
%     How long it took you to complete the assignment, and who did what. 

\section{Introduction}

\section{Motivation}

\section{Related Work}

\section{Algorithm/Technique}

\section{Results}

\section{Conclusions}

\newpage
\begin{thebibliography}{9}

% \bibitem{trees}
%   Bruneton, E.\& Neyret, F.
%   \emph{Real-time Realistic Rendering and Lighting of Forests}.
%   Computer Graphics Forum \textbf{31.2} (2012), available at
%   \url{http://hal.inria.fr/hal-00650120/en}.

% \bibitem{vecterrain}
%   Bruneton, E.\& Neyret, F.
%   \emph{Real-time rendering and editing of vector-based terrains}.
%   EUROGRAPHICS \textbf{27.2} (2008), available at
%   \url{http://www-ljk.imag.fr/Publications/Basilic/com.lmc.publi.PUBLI_Article@1222d975265_acc57d/article.pdf}.

% \bibitem{many_polygons}
%   Farid
%   \emph{"Yes, but how many polygons?" An artist blog entry with interesting numbers}
%   Beyond3D Forum (2007), available at
%   \url{http://forum.beyond3d.com/showthread.php?t=43975}.

% \bibitem{fastlightfield}
%   Kolb, A., Rezk-Salama, C. \& Todt, S.
%   \emph{Fast (Spherical) Light Field Rendering with Per-Pixel Depth}
%   University of Siegen (2007), available at
%   \url{http://www.cg.informatik.uni-siegen.de/data/Publications/2007/tr1107_LightField.pdf}.

% \bibitem{treeszbuf}
%   Max, N. \& Ohsaki, K.
%   \emph{Rendering Trees from Precomputed Z-Buffer Views}.
%   In Eurographics Rendering Workshop (1995), 45--54, available at
%   \url{http://citeseerx.ist.psu.edu/viewdoc/summary?doi=10.1.1.17.30}.
%   % (bibtex:)
%   % @INPROCEEDINGS{Max95renderingtrees,
%   %   author = {Nelson Max and Keiichi Ohsaki},
%   %   title = {Rendering Trees from Precomputed Z-Buffer Views},
%   %   booktitle = {In Eurographics Rendering Workshop},
%   %   year = {1995},
%   %   pages = {45--54}
%   % }

% Terrain generation is based on pseudocode from http://gameprogrammer.com/fractal.html
\end{thebibliography}

\end{document}