\documentclass{article}
\usepackage{fullpage}
\usepackage{hyperref}

\begin{document}
%Title
\title{Advanced Computer Graphics Final Project Proposal}
\author{Auston Sterling: sterla@rpi.edu \\ Brendon Justin: justib@rpi.edu}
\date{}
\maketitle

%Description
\section{Introduction}
For our final project, we plan to implement a portion of Eric Bruneton and Fabrice Neyret's algorithm for real-time realistic rendering and lighting of forests \cite{trees}. In particular, we plan on implementing the near-tree model, which precalculates two dimensional views of each tree, storing a selection of values per texel to assist in rendering. Once the simulation runs, these values are used to interpolate and determine how the tree should look from any given angle. We will also attempt to accurately and realistically light these trees from a number of factors. Getting this to look realistic and run in real-time could be a challenge, even with a source paper to refer to. Optimization will require use of hierarchical spatial data structures, storing data on the graphics card, and will likely require referring to other papers. Implementing every aspect of this method and optimizing it to run in real-time may be a bit too large for a final project, but there are many intermediate steps which could be stopped at while still having a good final project.

\section{Motivation}
Trees are often a plentiful part of landscapes, and so rendering them at the same level of fidelity as foreground objects can make real-time rendering impossible.  At the same time, lower fidelity objects can detract from a scene's appearance, even if they are part of the background.  

Quality tree models can be 250,000 or more triangles
(for example, see \url{http://www.sharecg.com/v/36271/gallery/5/3D-Model/Alder-Tree-OBJ?interstitial_displayed=Yes}).  Just 40 of such trees would involve 10 million polygons.  For comparison, a frame in the 2007 game Lost Planet can have 3 million polygons in a frame \cite{many_polygons}.  Clearly, convention rendering methods do not permit the use of such models in games.  Games instead include low quality tree models, or decrease landscape detail with distance from the player.

Bruneton and Fabrice \cite{trees} implemented a fast means of rendering high quality faraway trees.  We wish to duplicate their success, rendering 1,000 or more distant trees.  As in \cite{trees}, we plan to include complex lighting effects: intra-tree shadowing, self-shadowing and hotspotting for leaves, ‘silver-lining’, and ground shadowing.

%Some papers
\section{Research Papers}
The main source paper is \cite{trees}. We have identified a few relevant papers which may assist in the details of the implementation such as 
\cite{vecterrain}, which provides details about their quad-tree design \cite{treeszbuf}, an older paper describing a similar strategy for rendering trees, though with some ghosting and without as much attention given to lighting \cite{fastlightfield}, which was a large inspiration for the tree rendering algortihm.  However, the main paper provides plenty more references which could be useful. There will be plenty of reading to do.

%Intermediate steps
%\section{Intermediate Steps and Tasks}
%The first step is to create the many views of a model and simply use a nearest neighbor lookup to find the closest view and display it. This will likely have no lighting and will have extreme popping while rotating the camera. 
%After that, interpolation between views will probably be the highest priority. 
%From there, there are actually many different areas to pursue. We could take some steps to make the lighting more accurate on the different scales of the trees. There are many effects to take into account here, such as lighting from the sun and sky. 
%We could also attempt to implement the paper's system for storing and accessing the trees and terrain. A large part of the paper is solving the problem of rendering large amounts of densely packed trees, and it would be nice to make a forest look like a forest. 
%Another goal would be to optimize the program for higher frame rate. This could be done through a lot of methods, but it should be possible given that the original paper accomplished it. 
%If we had unlimited time, the goal would be to almost exactly replicate the process in the paper, but that is going to be beyond the scope of a one month project. We will likely plan as we go, deciding that maybe lighting is ``good enough for now'' and that optimization should take more of our efforts.


%Tasks
The tasks for the project have been grouped into multiple milestones for tracking progress.  Each task is also tentatively assigned to an individual:

\begin{itemize}
\item Milestone 1: TARGET-DATE-1
\begin{itemize}
\item Find tree model(s) (both)
\item Open and render tree model(s) (Auston)
\item Save views of rendered tree, with additional per-texel data ‘minimal and maximal depths $z$ and $\bar{z}$, an ambient occlusion $\delta$, and an opacity $\alpha$’\cite{trees} (Auston)
\item Render trees using nearest-neighbors to current angle (Brendon)
\end{itemize}

\item Milestone 2: TARGET-DATE-2
\begin{itemize}
\item Basic programmatic tree seeding (Brendon)
\item Basic terrain generation (Brendon)
\item Reconstruct 3d tree views using interpolation (Auston)
\end{itemize}

\item Milestone 3: DUE-DATE
\begin{itemize}
\item Lighting (see Lighting, below) (both)
\item Complex terrain generation and tree seeding (Brendon)
\item Optimization (Auston)
\end{itemize}

\item Lighting
\begin{itemize}
\item From the sun (both)
\begin{itemize}
\item Lighting on leaves, as a participating medium (both)
\begin{itemize}
\item Hotspotting at the level of groups of leaves (Brendon)
\item Hotspotting at the level of individual leaves (Auston)
\end{itemize}

\item Surface light scattering (Auston)
\end{itemize}

\item From the sky (Brendon)
\item On the ground (Brendon)
\end{itemize}

\end{itemize}

\begin {itemize}
\item The first milestone involves a majority of the precomputation for tree rendering. By TARGET-DATE-1, a grid of tree views should be rendered, though this will lack any lighting and will noticably pop between views while rotating the camera.

\item The second milestone introduces aspects of forests other than simply rendering a grid of trees. Having trees scattered over Earth-like terrain will help make the forest look realistic. At the same time, tree rendering can become more complete; removing the popping between views but still missing the lighting.

\item Milestone three finishes off the implementation with accurate lighting, as well as more detailed terrain and seeding. This is also the time to ensure that the program can run at at least a few frames per second. By the end of the project, fully lit and properly rendered trees should be dispersed over a realistic terrain.
\end {itemize}

\newpage
\begin{thebibliography}{9}

\bibitem{trees}
  Bruneton, E.\& Neyret, F.
  \emph{Real-time Realistic Rendering and Lighting of Forests}.
  Computer Graphics Forum \textbf{31.2} (2012), available at
  \url{http://hal.inria.fr/hal-00650120/en}.

\bibitem{vecterrain}
  Bruneton, E.\& Neyret, F.
  \emph{Real-time rendering and editing of vector-based terrains}.
  EUROGRAPHICS \textbf{27.2} (2008), available at
  \url{http://www-ljk.imag.fr/Publications/Basilic/com.lmc.publi.PUBLI_Article@1222d975265_acc57d/article.pdf}.

\bibitem{many_polygons}
  Farid
  \emph{"Yes, but how many polygons?" An artist blog entry with interesting numbers}
  Beyond3D Forum (2007), available at
  \url{http://forum.beyond3d.com/showthread.php?t=43975}.

\bibitem{fastlightfield}
  Kolb, A., Rezk-Salama, C. \& Todt, S.
  \emph{Fast (Spherical) Light Field Rendering with Per-Pixel Depth}
  University of Siegen (2007), available at
  \url{http://www.cg.informatik.uni-siegen.de/data/Publications/2007/tr1107_LightField.pdf}.

\bibitem{treeszbuf}
  Max, N. \& Ohsaki, K.
  \emph{Rendering Trees from Precomputed Z-Buffer Views}.
  In Eurographics Rendering Workshop (1995), 45--54, available at
  \url{http://citeseerx.ist.psu.edu/viewdoc/summary?doi=10.1.1.17.30}.
  % (bibtex:)
  % @INPROCEEDINGS{Max95renderingtrees,
  %   author = {Nelson Max and Keiichi Ohsaki},
  %   title = {Rendering Trees from Precomputed Z-Buffer Views},
  %   booktitle = {In Eurographics Rendering Workshop},
  %   year = {1995},
  %   pages = {45--54}
  % }

\end{thebibliography}

\end{document}
